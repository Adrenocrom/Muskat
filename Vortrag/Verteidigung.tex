\documentclass[xcolor=dvipsnames]{beamer}
\usepackage[utf8]{inputenc}
\usepackage[T1]{fontenc}
\usepackage{lmodern}
\usepackage{xcolor}
\usepackage{color}
\usepackage{enumitem}
\usepackage{tikz}

\setlist[itemize,1]{label=\textbullet}
\setlist[itemize,2]{label=$-$}

%\usebackgroundtemplate{\includegraphics[width=\paperwidth,height=\paperheight]{images/forma4.png}}
\setbeamercolor{frametitle}{bg=Gray, fg=white}
\setbeamercolor{normal text}{fg=black}

\setbeamercolor{title}{fg=white}
\setbeamercolor{titlelike}{fg=white}
\setbeamercolor{structure}{bg=Gray, fg=black}
\useoutertheme{shadow}
%\useinnertheme{rounded}

\beamertemplatenavigationsymbolsempty

\setbeamertemplate{footline}[frame number] 
\setbeamertemplate{headline}{}

%\renewcommand*{\bibfont}{\footnotesize}
\bibliographystyle{plain}

\title{Optimierung und Übertragung von Tiefengeometrie für Remote-Rendering}
\author{Josef Schulz}
\institute{Technische Universität Dresden}

\begin{document}

\begin{frame}
	\maketitle
	\nocite*{}
	\thispagestyle{empty}
\end{frame}

\begin{frame}
	\frametitle{Aufgabenstellung}
	
	Ziel der Arbeit:	
	
	\vspace*{0.4cm}
	
	\begin{itemize}
		\setlength{\itemsep}{8pt}
		
		\item Entwicklung eines CPU-Renderers für Partikeldaten
		\item basierend auf dem Emissions- und Absorptionsmodell
		\item Unterstützung von beliebig vielen Punkt und Richtungslichtquellen
		\item Beschränkung auf Kugelglyphen
	\end{itemize}
	
	\vspace*{0.2cm}
	
	\begin{itemize}
		\item Genauigkeit wird Geschwindigkeit übergeordnet
		\item liefert Ground-Truth-Bilder
	\end{itemize}
\end{frame}

\begin{frame}
	\frametitle{frame title text}
	\tableofcontents
\end{frame}

\section{bla}
\begin{frame}
\frametitle{bla}
hj
\end{frame}

{\tiny\bibliography{literatur}}

\end{document}
