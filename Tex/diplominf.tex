\documentclass[hyperref,german,diplominf]{cgvpub}
%weitere Optionen zum Erg�nzen (in eckigen Klammern):
% 
% bibnum	numerische Literaturschl�ssel
% final 	f�r Abgabe	
% lof			Abbildungsverzeichis
% lot			Tabellenverzeichnis
% noproblem	keine Aufgabenstellung
% notoc			kein Inhaltsverzeichnis
% twoside		zweiseitig

\author{Josef Schulz}
\title{Optimierung und �bertragung von Tiefengeometrie f�r Remote-Visualisierung}
\birthday{20. Oktober 1989}
\placeofbirth{Naumburg (Saale)}
\matno{3658867}
\betreuer{Dr. Sebastian Grottel}
\bibfiles{literatur}

\problem{
In Big-Data-Szenarien in der Visualisierung spielt der Ansatz der Remote-Visualisierung eine zunehmende Rolle.  
Moderne Netzwerktechnologien bieten gro�e Daten�bertragungsraten und niedrige Latenzzeiten. F�r die 
interaktive Visualisierung sind aber selbst kleinste Latenzzeiten problematisch. Um diese vor dem Benutzer maskieren zu k�nnen, kann eine Extrapolation der Darstellung durchgef�hrt. 
Diese Berechnungen erfordern zus�tzlich zum normalen Farbbild weitere Daten, beispielsweise 
ein Tiefenbild und die Daten der verwendeten Kameraeinstellung.
F�r die Darstellungsextrapolation werden Farb- und Tiefenbild zusammen interpretiert, beispielsweise als Punktwolke oder H�henfeldgeometrie. 
Im Rahmen dieser Arbeit soll untersucht werden, wie die Darstellung mittels H�henfeldgeometrie optimiert  werden kann. 
Ans�tze sind hierf�r Algorithmen aus der Netzvereinfachung. Zu erwarten sind sowohl 
harte Kanten als auch glatte Verl�ufe der Tiefenwerte, welche sich in der Netzgeometrie durch 
adaptive Vernetzung mit reduziertem Datenaufwand darstellen lassen.


Dem Szenario der Web-basierten Remote-Visualisierung folgend soll der Web-Browser als
Klient-Komponente eingesetzt werden. Die einzusetzenden Technologien sind HTML5, Javascript, 
WebGL und WebSockets. Entsprechende Javascript-Bibliotheken sollen genutzt 
werden um die Qualit�t und Wartbarkeit des Quellcodes zu steigern. F�r die Server-Komponente darf die Technologie vom Bearbeiter frei gew�hlt werden.

Zu Beginn der Arbeit wird eine Literatur-Recherche zu Web-basierter Visualisierung und Remote-Visualisierung erfolgen. 
Schwerpunkte  sind hierbei  die  Bild-Extrapolation, Vernetzung 
und Rekonstruktion auf Basis von Tiefenbildern und die Netzoptimierung und -Vereinfachung. 
Im Anschluss an die Literaturrecherche wird ein Konzept f�r die Implementierung mit dem 
Betreuer abgesprochen und anschlie�end als prototypische Software umgesetzt. Folgendes 
Szenario dient als Grundlage f�r dieses Konzept:

Als  Eingabedaten  stehen  mehrere  Datens�tze aus  unterschiedlichen  Szenarien  der  wissenschaftlichen Visualisierung zur Verf�gung. F�r jeden Datensatz sind mehrere Tripel aus Farbbild, Tiefenbild und Kamera-Parameter gegeben.
Die Serverkomponente bereitet einen Datensatz auf und bietet ihn dem Klienten an. Diese Aufbereitung ist vor allem die Generierung einer optimierten  Tiefennetzgeometrie  aus  den  Tiefenbilddaten.  Der  Klient  fordert  Farbbilder,  Kameraeinstellungen und Tiefengeometrie von Tripel-Paaren an.
Konzeptuell wird ein Tripel als aktueller Zustand und das zweite Tripel als Ground-Truth einer Bildextrapolation verstanden. 
Diese k�nnen daher auch in dieser Reihenfolge angefordert werden. 
Die Tripel werden zwischen  Klient  und  Server  direkt  per  Sockets/WebSockets  �bertragen.
Die Daten des ersten Tripels werden anschlie�end genutzt um dessen Farbbild in die Ansicht des zweiten Tripels extrapoliert. Hierbei werden vom zweiten Tripel nur die Kameraeinstellung genutzt.
Diese Extrapolation wird  Klient-seitig in WebGL implementiert  damit  alle  Berechnungen  auf  der  GPU 
ausgef�hrt werden. 
Anschlie�end wird das extrapolierte Bild mit dem originalen Ground-Truth-Farbbild  aus  dem  zweiten  Tripel  verglichen  um  die  Qualit�t  der  Extrapolation  zu  bewerten, z.B. durch SSIM.

Die umgesetzte L�sung wird ausf�hrlich evaluiert.
Zentraler Wert ist hierbei die Bildqualit�t nach der Extrapolation abh�ngig vom Winkelunterschied zwischen den Kameraeinstellungen und den Parametern der Vereinfachung der Tiefennetzgeometrie. 
Hierf�r werden Tripel-Paare aus den Datens�tzen und Variationen der Parameter der Algorithmen systematisch 
und automatisiert vermessen. Untersuchungen zum Laufzeitverhalten der Netzoptimierung im Server 
und der Bildextrapolation im Klienten sind optional durchzuf�hren.
}

\copyrighterklaerung{Hier soll jeder Autor die von ihm eingeholten
Zustimmungen der Copyright-Besitzer angeben bzw. die in Web Press
Rooms angegebenen generellen Konditionen seiner Text- und
Bild"ubernahmen zitieren.}
\acknowledgments{Die Danksagung...}
\abstracten{abstract text english}
\abstractde{ Zusammenfassung Text Deutsch}

\usepackage{pgfplots}
\usepackage{filecontents}

\input{../results/2/512x512_Delaunay/D8/L1.0/I0.0/results.tex}
\input{../results/2/512x512_Delaunay/D8/L1.0/I0.5/results.tex}
\input{../results/2/512x512_Delaunay/D8/L1.0/I1.0/results.tex}

\documentclass{article}
\usepackage{pgfplots}
\usepackage{filecontents}

\input{res/results.tex}

\begin{document}
\begin{tikzpicture}
\begin{axis}[axis lines = middle, enlargelimits = true]
\addplot[blue, only marks, mark=x] table [x=a, y=p, col sep=comma] {div_data.csv};
\end{axis}
\end{tikzpicture}

\begin{tikzpicture}
\begin{axis}[axis lines = middle, enlargelimits = true]
\addplot[blue, only marks, mark=x] table [x=a, y=p, col sep=comma] {div_data_mean.csv};
\end{axis}
\end{tikzpicture}

\begin{tikzpicture}
\begin{axis}[axis lines = middle, enlargelimits = true]
\addplot[blue, only marks, mark=x] table [x=a, y=m, col sep=comma] {div_data_mean.csv};
\end{axis}
\end{tikzpicture}


\vspace*{1cm}

\begin{tikzpicture}
\begin{axis}[axis lines = middle, enlargelimits = true]
\addplot[red, only marks, mark=x] table [x=a, y=r, col sep=comma] {div_data.csv};
\addplot[green, only marks, mark=x] table [x=a, y=g, col sep=comma] {div_data.csv};
\addplot[blue, only marks, mark=x] table [x=a, y=b, col sep=comma] {div_data.csv};
\addplot[black, only marks, mark=x] table [x=a, y=m, col sep=comma] {div_data.csv};
\end{axis}
\end{tikzpicture}
\end{document}


\begin{document}

\chapter{Einleitung}



\chapter{Verwandte Arbeiten}
\chapter{Ergebnisse}

\begin{figure}[htbp]
	\centering
		\includegraphics{../results/2/512x512_Delaunay/D8/L1.0/I1.0/delaunay.png}
		\includegraphics{../results/2/512x512_Delaunay/D8/L0.5/I0.2/delaunay.png}
	\caption{beschriftung}
	\label{fig:diplominf}
\end{figure}

\chapter{Noch mehr Ergebnisse}

\begin{tikzpicture}
\begin{axis}[axis lines = middle, enlargelimits = true]
\addplot[blue, only marks, mark=x] table [x=a, y=p, col sep=comma] {div_data_4_512x512D8L1.0I1.0.csv};
\end{axis}
\end{tikzpicture}

\begin{tikzpicture}
\begin{axis}[axis lines = middle, enlargelimits = true]
\addplot[blue, only marks, mark=x] table [x=a, y=p, col sep=comma] {div_data_mean_4_512x512D8L1.0I1.0.csv};
\end{axis}
\end{tikzpicture}

\begin{tikzpicture}
\begin{axis}[axis lines = middle, enlargelimits = true]
\addplot[green, mark=x] table [x=a, y=m, col sep=comma] {div_data_mean_4_512x512D8L1.0I0.0.csv};
\addplot[red, mark=x] table [x=a, y=m, col sep=comma] {div_data_mean_4_512x512D8L1.0I0.5.csv};
\addplot[blue, mark=x] table [x=a, y=m, col sep=comma] {div_data_mean_4_512x512D8L1.0I1.0.csv};
\end{axis}
\end{tikzpicture}


\vspace*{1cm}

\begin{tikzpicture}
\begin{axis}[axis lines = middle, enlargelimits = true]
\addplot[red, only marks, mark=x] table [x=a, y=r, col sep=comma] {div_data_4_512x512D8L1.0I1.0.csv};
\addplot[green, only marks, mark=x] table [x=a, y=g, col sep=comma] {div_data_4_512x512D8L1.0I1.0.csv};
\addplot[blue, only marks, mark=x] table [x=a, y=b, col sep=comma] {div_data_4_512x512D8L1.0I1.0.csv};
\addplot[black, only marks, mark=x] table [x=a, y=m, col sep=comma] {div_data_4_512x512D8L1.0I1.0.csv};
\end{axis}
\end{tikzpicture}

\begin{tikzpicture}
\begin{axis}[axis lines = middle, enlargelimits = true]
\addplot[green] table [x=i, y=min, col sep=comma]{div_duration_info_4_512x512D8L1.0I1.0.csv};
\addplot[red] table [x=i, y=max, col sep=comma]{div_duration_info_4_512x512D8L1.0I1.0.csv};
\addplot[black] table [x=i, y=mean, col sep=comma]{div_duration_info_4_512x512D8L1.0I1.0.csv};1
\addplot[blue, mark=x] table [x=i, y=d, col sep=comma]{div_duration_info_4_512x512D8L1.0I1.0.csv};
\end{axis}
\end{tikzpicture}

\begin{tikzpicture}
    \begin{axis}[view={0}{90}, 
    			 grid=major, 
    			 colormap={greenyellow}{
    			 	rgb255(0cm)	 =(128,0,0)
    			 	rgb255(0.5cm) =(255,255,0) 
    			 	rgb255(1cm) =(0,200,0)
    			 }, 
    			 colorbar, 
    			 enlargelimits=true, 
    			 xtick={0.0,0.1,...,1.1},
        		 ytick={0.0,0.1,...,1.1},
        		 xmajorgrids=true,
        		 ymajorgrids=true]
        
        \addplot3 [only marks, 
        		   mark=square*, 
        		   mark size=7, 
        		   ycomb, 
        		   scatter] file {div.csv};
        		  % scatter/use mapped color={
            	%		draw=mapped color,
            	%		fill=mapped color,
          		 %  },] file {div.csv};
        
    \end{axis}
\end{tikzpicture}

\cite*{}
\end{document}